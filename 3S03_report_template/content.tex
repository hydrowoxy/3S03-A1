\section{Testing with generative AI}
\subsection{a)} 

I asked ChatGPT 5.2 to specify three types of tests for the game described above.

\vspace{20pt}

The specific prompt I used was:

\vspace{10pt}

Suppose I am developing an action-adventure game such that:
a) players explore a post-apocalyptic version of McMaster University; b) up to eight players 
may be accosted by zombies, mutated professors, angry Deans, killer squirrels, and more; c) 
gameplay can be both competitive and cooperative; and d) the game will be released on the 
PS5, Xbox Series X, Nintendo Switch and Sega Genesis (which is 16-bit and hasn't been 
manufactured since 1997). 

Specify three types of tests for such a game.

\vspace{20pt}

The response I recieved was as follows (it originally included more emojis and markdown formatting,
which I removed (and re-formatted) so that my document would compile and be legible):

\vspace{10pt}

For a game like this — especially one that spans \textbf{modern consoles and a 16-bit relic like the Sega Genesis} — testing needs to cover functionality, performance, and platform constraints very carefully.

Here are \textbf{three distinct and appropriate types of testing}:

1. Functional Testing (Gameplay \& Mechanics)

\textbf{Purpose:} Ensure the game works according to design.

This would verify:

\begin{itemize}
    \item Players can explore the post-apocalyptic McMaster campus without falling through geometry
    \item Zombies, mutated professors, angry Deans, and killer squirrels spawn correctly
    \item Combat, health, inventory, and abilities function as intended
    \item Competitive and cooperative modes behave correctly
    \item Up to eight players connect and interact properly
\end{itemize}

Examples of functional test cases:

\begin{itemize}
    \item If 8 players join a session, all characters spawn correctly.
    \item If a Dean is defeated, loot drops trigger.
    \item Friendly fire behaves correctly depending on mode (co-op vs competitive).
    \item Save/load restores correct state.
\end{itemize}

This ensures the \textbf{core gameplay logic works}.

2. Performance \& Stress Testing (Multiplayer \& AI Load)

\textbf{Purpose:} Ensure the game runs smoothly under load.

Your game supports:

\begin{itemize}
    \item Up to 8 players
    \item Multiple enemy types
    \item Large explorable campus
    \item Both modern and legacy hardware
\end{itemize}

Testing would include:

\begin{itemize}
    \item Frame rate stability (60 FPS on PS5/Xbox, acceptable performance on Switch)
    \item Mass zombie swarm scenarios
    \item Network latency under 8-player online sessions
    \item Memory usage limits
\end{itemize}

Especially important:

\begin{itemize}
    \item The \textbf{Nintendo Switch} has weaker hardware.
    \item The \textbf{Sega Genesis} has extreme memory and CPU constraints (16-bit, limited RAM, cartridge storage).
\end{itemize}

Stress testing would reveal:

\begin{itemize}
    \item Frame drops when 50 squirrels attack simultaneously
    \item Network desync in competitive mode
    \item AI pathfinding breakdowns under heavy load
\end{itemize}

3. Compatibility \& Platform Testing (Cross-Console + 16-bit Constraints)

\textbf{Purpose:} Ensure the game works correctly across very different hardware environments.

You are releasing on:

\begin{itemize}
    \item PlayStation 5
    \item Xbox Series X
    \item Nintendo Switch
    \item Sega Genesis
\end{itemize}

Key concerns:

Modern Consoles (PS5 / Xbox Series X)

\begin{itemize}
    \item 4K rendering compatibility
    \item Ray tracing support (if implemented)
    \item SSD load speed optimization
    \item Controller haptics
\end{itemize}

Nintendo Switch

\begin{itemize}
    \item Lower resolution scaling
    \item Docked vs handheld mode behavior
    \item Limited GPU power
\end{itemize}

Sega Genesis

\begin{itemize}
    \item 16-bit graphics limitations
    \item Severe RAM restrictions
    \item Cartridge size constraints
    \item No modern multiplayer networking
    \item Likely redesigned gameplay mechanics
\end{itemize}

Compatibility testing ensures:

\begin{itemize}
    \item UI scaling works at different resolutions
    \item Input mapping works on each controller
    \item Save systems comply with platform standards
    \item The Genesis version doesn’t exceed hardware limits
\end{itemize}

Summary

The three appropriate testing types are:

\begin{enumerate}
    \item \textbf{Functional Testing} – Does the game behave correctly?
    \item \textbf{Performance \& Stress Testing} – Does it run smoothly under load?
    \item \textbf{Compatibility/Platform Testing} – Does it work properly across PS5, Xbox, Switch, and Genesis?
\end{enumerate}

Given the extreme hardware diversity (especially including a 1990s 16-bit console), 
\textbf{platform compatibility testing would be especially critical} for this project.

\subsection{b)} Two poor, incorrect, or unclear aspects of the response are:

\begin{enumerate}
\item Tests are categorized by type (functional, performance, compatibility) 
but not granularity (unit, integration, system).

We have seen in class that unit and integration testing are often conflated. This is
bad because the distinction is important for understanding feedback speed, debugging 
difficulty, and test design. The AI response not only fails to make this distinction,
but also fails to even mention unit testing; it is unclear whether the tests it suggests
are to be implemented as unit, integration or system tests (though it seems the AI is
relying only on system-level tests based on the descriptions provided). This is a 
significant omission, as unit tests are crucial for verifying the correctness of 
individual components in isolation.

\item No measurable adequacy criteria are provided for any of the tests.

Adequacy criteria such as statement, branch, or MC/DC coverage are completely omitted. 
Coverage is defined as a way to measure a proportion of the structure or domain that a 
program, test case, or test suite exercises: this helps us understand when testing is
sufficient. The AI response provides no guidance on how to determine when testing is
adequate. 

\end{enumerate}




\section{Testing with Junit}

\section{Testing parts of large systems}

\section{Test driven development (TDD)}

\section{Test coverage and AI}